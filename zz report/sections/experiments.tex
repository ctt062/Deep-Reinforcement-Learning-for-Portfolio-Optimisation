\section{Experimental Setup}
\label{sec:experiments}

\subsection{Asset Universe}

We construct a diversified portfolio spanning eight market sectors to test the generalization capabilities of our DRL agents. Table \ref{tab:assets} presents the complete asset universe.

\begin{table}[htbp]
\centering
\caption{Asset universe with 18 diversified instruments}
\label{tab:assets}
\begin{tabular}{@{}lll@{}}
\toprule
\textbf{Ticker} & \textbf{Name} & \textbf{Sector} \\
\midrule
AAPL & Apple Inc. & Technology \\
MSFT & Microsoft Corporation & Technology \\
GOOGL & Alphabet Inc. & Technology \\
NVDA & NVIDIA Corporation & Technology \\
AMZN & Amazon.com Inc. & Technology \\
\midrule
JNJ & Johnson \& Johnson & Healthcare \\
UNH & UnitedHealth Group & Healthcare \\
PFE & Pfizer Inc. & Healthcare \\
\midrule
JPM & JPMorgan Chase & Financials \\
V & Visa Inc. & Financials \\
\midrule
WMT & Walmart Inc. & Consumer Staples \\
COST & Costco Wholesale & Consumer Staples \\
\midrule
SPY & S\&P 500 ETF & Index \\
QQQ & NASDAQ-100 ETF & Index \\
IWM & Russell 2000 ETF & Index \\
\midrule
TLT & 20+ Year Treasury ETF & Bonds \\
AGG & Aggregate Bond ETF & Bonds \\
\midrule
GLD & Gold ETF & Commodities \\
\bottomrule
\end{tabular}
\end{table}

The asset selection provides:
\begin{itemize}
    \item \textbf{Sector Diversification}: Eight distinct sectors reduce concentration risk
    \item \textbf{Asset Class Diversity}: Equities, bonds, and commodities offer different risk-return profiles
    \item \textbf{Liquidity}: All assets are highly liquid with minimal transaction costs
    \item \textbf{Data Quality}: Long history of reliable price data available
\end{itemize}

\subsection{Data Period and Split}

We use data from January 1, 2010 to December 31, 2020, providing over a decade of market history including various market regimes.

\begin{table}[htbp]
\centering
\caption{Data period configuration}
\label{tab:periods}
\begin{tabular}{@{}llll@{}}
\toprule
\textbf{Period} & \textbf{Start} & \textbf{End} & \textbf{Purpose} \\
\midrule
Training & 2010-01-01 & 2018-12-31 & Agent learning \\
Testing & 2019-01-01 & 2020-12-31 & Out-of-sample evaluation \\
\bottomrule
\end{tabular}
\end{table}

Key characteristics of each period:

\textbf{Training Period (2010-2018):}
\begin{itemize}
    \item Post-financial crisis recovery (2010-2012)
    \item Quantitative easing era (2012-2015)
    \item Low volatility bull market (2016-2018)
    \item Various market corrections and sector rotations
    \item Approximately 2,265 trading days
\end{itemize}

\textbf{Testing Period (2019-2020):}
\begin{itemize}
    \item 2019: Strong bull market with trade war uncertainties
    \item 2020: COVID-19 pandemic crash (March) and subsequent recovery
    \item Extreme volatility regime (VIX spike to 82.69 in March 2020)
    \item V-shaped recovery demonstrating market resilience
    \item Approximately 504 trading days
\end{itemize}

\subsection{Why No Validation Set?}

Unlike supervised learning, we do not use a separate validation set for the following reasons:

\begin{enumerate}
    \item \textbf{Sequential Data}: Financial time series must maintain temporal order; shuffling would create look-ahead bias.
    
    \item \textbf{On-Policy Learning}: Agents learn from their own experience in the environment, making traditional validation less applicable.
    
    \item \textbf{Hyperparameter Selection}: We use established hyperparameters from literature rather than extensive tuning on validation data.
    
    \item \textbf{Overfitting Prevention}: Early stopping based on training reward curves and model capacity constraints prevent overfitting.
    
    \item \textbf{Maximum Training Data}: All available pre-2019 data is used for training to maximize learning from diverse market conditions.
\end{enumerate}

\subsection{Hyperparameter Configuration}

\subsubsection{DDPG Hyperparameters}

\begin{table}[htbp]
\centering
\caption{DDPG hyperparameter configuration}
\label{tab:ddpg_params}
\begin{tabular}{@{}ll@{}}
\toprule
\textbf{Parameter} & \textbf{Value} \\
\midrule
Learning rate (actor) & $1 \times 10^{-4}$ \\
Learning rate (critic) & $1 \times 10^{-3}$ \\
Replay buffer size & 100,000 \\
Batch size & 128 \\
Discount factor ($\gamma$) & 0.99 \\
Soft update coefficient ($\tau$) & 0.005 \\
Network architecture & [256, 256] \\
Activation function & ReLU \\
OU noise $\sigma$ & 0.1 \\
OU noise $\theta$ & 0.15 \\
Training timesteps & 200,000 \\
\bottomrule
\end{tabular}
\end{table}

\subsubsection{PPO Hyperparameters}

\begin{table}[htbp]
\centering
\caption{PPO hyperparameter configuration}
\label{tab:ppo_params}
\begin{tabular}{@{}ll@{}}
\toprule
\textbf{Parameter} & \textbf{Value} \\
\midrule
Learning rate & $3 \times 10^{-4}$ \\
Steps per update & 2,048 \\
Batch size & 64 \\
Number of epochs & 10 \\
Discount factor ($\gamma$) & 0.99 \\
GAE parameter ($\lambda$) & 0.95 \\
Clip range ($\epsilon$) & 0.2 \\
Entropy coefficient & 0.01 \\
Value function coefficient & 0.5 \\
Network architecture & [256, 256] \\
Training timesteps & 200,000 \\
\bottomrule
\end{tabular}
\end{table}

\subsubsection{Environment Configuration}

\begin{table}[htbp]
\centering
\caption{Environment configuration parameters}
\label{tab:env_params}
\begin{tabular}{@{}ll@{}}
\toprule
\textbf{Parameter} & \textbf{Value} \\
\midrule
Initial portfolio value & \$1,000,000 \\
Transaction cost & 0.001 (10 bps) \\
Lookback window & 20 days \\
Risk-free rate & 2\% annual \\
Risk penalty coefficient & 0.5 \\
Maximum hedge ratio & 0.2 \\
Put option strike & 95\% of portfolio value \\
Put option expiry & 30 days \\
Stop-loss thresholds & [5\%, 10\%, 15\%] \\
\bottomrule
\end{tabular}
\end{table}

\subsection{Benchmark Strategies}

We compare DRL agents against several benchmark strategies:

\begin{enumerate}
    \item \textbf{Equal Weight (1/N)}: Allocates equal weight to all assets
    \begin{equation}
    w_i = \frac{1}{n} \quad \forall i
    \end{equation}
    
    \item \textbf{SPY Buy-and-Hold}: 100\% allocation to S\&P 500 ETF
    \begin{equation}
    w_{\text{SPY}} = 1, \quad w_i = 0 \; \forall i \neq \text{SPY}
    \end{equation}
    
    \item \textbf{60/40 Portfolio}: Traditional balanced allocation
    \begin{equation}
    w_{\text{equity}} = 0.6, \quad w_{\text{bonds}} = 0.4
    \end{equation}
\end{enumerate}

\subsection{Evaluation Metrics}

We evaluate performance using multiple metrics:

\begin{table}[htbp]
\centering
\caption{Performance evaluation metrics}
\label{tab:metrics}
\begin{tabular}{@{}lp{8cm}@{}}
\toprule
\textbf{Metric} & \textbf{Description} \\
\midrule
Total Return & Cumulative portfolio return over test period \\
Annualized Return & Geometric mean annual return \\
Annualized Volatility & Standard deviation of returns, annualized \\
Sharpe Ratio & Risk-adjusted return measure \\
Sortino Ratio & Downside risk-adjusted return \\
Maximum Drawdown & Largest peak-to-trough decline \\
Calmar Ratio & Annual return divided by max drawdown \\
Win Rate & Percentage of positive return days \\
Average Daily Return & Mean daily portfolio return \\
Options P\&L & Cumulative profit/loss from hedging \\
\bottomrule
\end{tabular}
\end{table}

\subsection{Computational Environment}

Experiments were conducted using the following setup:

\begin{itemize}
    \item \textbf{Hardware}: MacBook Pro with Apple M-series chip
    \item \textbf{Software}: Python 3.13.3, PyTorch 2.x
    \item \textbf{Libraries}: 
    \begin{itemize}
        \item Stable-Baselines3 for DRL implementations
        \item Gymnasium for environment interface
        \item NumPy, Pandas for data manipulation
        \item Matplotlib, Seaborn for visualization
    \end{itemize}
    \item \textbf{Training Time}: Approximately 30-60 minutes per agent
\end{itemize}

\subsection{Reproducibility}

To ensure reproducibility:
\begin{itemize}
    \item Random seeds are fixed across all experiments
    \item Configuration files specify all hyperparameters
    \item Data preprocessing steps are documented
    \item Trained models are saved and versioned
    \item Evaluation scripts produce deterministic results
\end{itemize}

The complete codebase is available at:\\
\url{https://github.com/[repository-link]}
