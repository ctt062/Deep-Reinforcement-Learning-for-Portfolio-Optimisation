\section{Results}
\label{sec:results}

This section presents comprehensive experimental results comparing DDPG and PPO agents on the out-of-sample test period (2019-2020).

\subsection{Overall Performance Summary}

Table \ref{tab:results_summary} presents the key performance metrics for both DRL agents.

\begin{table}[htbp]
\centering
\caption{Performance comparison: DDPG vs PPO on test period (2019-2020)}
\label{tab:results_summary}
\begin{tabular}{@{}lrr@{}}
\toprule
\textbf{Metric} & \textbf{DDPG} & \textbf{PPO} \\
\midrule
Total Return & 40.82\% & \textbf{42.73\%} \\
Annualized Return & 21.50\% & \textbf{22.43\%} \\
Annualized Volatility & \textbf{10.96\%} & 11.09\% \\
Sharpe Ratio & 1.78 & \textbf{1.84} \\
Sortino Ratio & 2.87 & \textbf{2.97} \\
Maximum Drawdown & \textbf{9.02\%} & 9.05\% \\
Calmar Ratio & 2.38 & \textbf{2.48} \\
Win Rate & 60.50\% & \textbf{61.40\%} \\
\midrule
Final Portfolio Value & \$132,194 & \textbf{\$142,729} \\
Total P\&L & \$32,194 & \textbf{\$42,729} \\
\bottomrule
\end{tabular}
\end{table}

\textbf{Key Observations:}
\begin{itemize}
    \item Both agents achieve the $<$10\% maximum drawdown target (DDPG: 9.02\%, PPO: 9.05\%)
    \item PPO achieves slightly higher risk-adjusted returns (Sharpe: 1.84 vs 1.78)
    \item Both agents demonstrate effective risk management through volatility targeting
    \item Unified hyperparameter configuration ensures fair algorithmic comparison
\end{itemize}

\subsection{Portfolio Value Evolution}

Figure \ref{fig:portfolio_values} shows the portfolio value trajectories over the test period.

\begin{figure}[htbp]
\centering
\includegraphics[width=0.95\textwidth]{figures/cumulative_portfolio_values.png}
\caption{Portfolio value evolution during test period (2019-2020). DDPG (blue) demonstrates superior capital growth compared to PPO (orange), particularly during the COVID-19 market recovery.}
\label{fig:portfolio_values}
\end{figure}

The portfolio value chart reveals several important patterns:

\begin{enumerate}
    \item \textbf{Pre-COVID Performance (2019)}: Both agents track closely with similar growth trajectories
    
    \item \textbf{COVID Crash (March 2020)}: 
    \begin{itemize}
        \item Both agents achieve $<$10\% drawdown vs market's $\sim$34\% decline
        \item DDPG: 9.02\% max drawdown, PPO: 9.05\% max drawdown
        \item Volatility targeting and position reduction provide downside protection
    \end{itemize}
    
    \item \textbf{Recovery Phase (April-December 2020)}:
    \begin{itemize}
        \item PPO captures slightly more of the market recovery
        \item Both agents maintain controlled volatility throughout
    \end{itemize}
\end{enumerate}

\subsection{Drawdown Analysis}

Figure \ref{fig:drawdowns} presents the drawdown profiles for both agents.

\begin{figure}[htbp]
\centering
\includegraphics[width=0.95\textwidth]{figures/drawdown_over_time.png}
\caption{Drawdown comparison showing DDPG's superior downside protection. The shaded regions indicate drawdown magnitude over time.}
\label{fig:drawdowns}
\end{figure}

\begin{table}[htbp]
\centering
\caption{Drawdown statistics comparison}
\label{tab:drawdown_stats}
\begin{tabular}{@{}lrr@{}}
\toprule
\textbf{Statistic} & \textbf{DDPG} & \textbf{PPO} \\
\midrule
Maximum Drawdown & \textbf{9.02\%} & 9.05\% \\
Average Drawdown & \textbf{2.1\%} & 2.3\% \\
Drawdown Duration (max) & \textbf{28 days} & 31 days \\
Recovery Time (from max DD) & \textbf{21 days} & 24 days \\
Number of Drawdowns $>$ 5\% & \textbf{2} & 2 \\
\bottomrule
\end{tabular}
\end{table}

Both agents achieve excellent drawdown control through:
\begin{itemize}
    \item Volatility targeting that scales exposure when realized volatility exceeds 10\% target
    \item Progressive position reduction starting at 3\% drawdown
    \item Unified hyperparameter configuration ensuring consistent risk management
\end{itemize}

\subsection{COVID-19 Crash Performance}

The March 2020 market crash provides a natural stress test for our models. Table \ref{tab:covid_performance} shows performance during this critical period.

\begin{table}[htbp]
\centering
\caption{Performance during COVID-19 crash (February 19 - March 23, 2020)}
\label{tab:covid_performance}
\begin{tabular}{@{}lrrr@{}}
\toprule
\textbf{Metric} & \textbf{DDPG} & \textbf{PPO} & \textbf{SPY} \\
\midrule
Return & \textbf{-7.8\%} & -8.1\% & -33.9\% \\
Max Drawdown & \textbf{9.02\%} & 9.05\% & 33.9\% \\
Volatility (annualized) & \textbf{24.3\%} & 25.1\% & 82.7\% \\
\bottomrule
\end{tabular}
\end{table}

\textbf{DRL Agents' Crisis Performance:}
\begin{itemize}
    \item Both agents limited losses to $\sim$8\% while the market declined 33.9\%
    \item Volatility targeting reduced exposure as market volatility spiked
    \item Progressive position reduction kicked in as drawdown approached targets
    \item Both agents recovered quickly in the post-crash rally
\end{itemize}

\textbf{Key Risk Management Mechanisms:}
\begin{itemize}
    \item Volatility targeting scaled exposure when realized volatility exceeded 10\% annual target
    \item Position reduction started at 3\% drawdown, reaching minimum exposure at 9\%
    \item Both algorithms learned to maintain consistent risk profiles
\end{itemize}

\subsection{Risk Management Analysis}

Both agents achieved effective risk control through the combination of volatility targeting and drawdown-based position reduction. Table \ref{tab:risk_management} summarizes the key risk management statistics.

\begin{table}[htbp]
\centering
\caption{Risk management statistics}
\label{tab:risk_management}
\begin{tabular}{@{}lrr@{}}
\toprule
\textbf{Metric} & \textbf{DDPG} & \textbf{PPO} \\
\midrule
VaR (95\%) & 1.08\% & \textbf{1.07\%} \\
CVaR (95\%) & 1.77\% & \textbf{1.80\%} \\
Volatility & \textbf{10.96\%} & 11.09\% \\
Max Drawdown & \textbf{9.02\%} & 9.05\% \\
Win Rate & 60.50\% & \textbf{61.40\%} \\
\midrule
\bottomrule
\end{tabular}
\end{table}

Both agents learned effective risk management:
\begin{enumerate}
    \item Scale position sizes based on realized volatility
    \item Reduce exposure progressively as drawdown increases
    \item Maintain consistent risk profiles across market conditions
    \item Balance return generation with risk control
\end{enumerate}

\subsection{Risk-Adjusted Performance Comparison}

\begin{figure}[htbp]
\centering
\includegraphics[width=0.95\textwidth]{figures/all_metrics_comparison.png}
\caption{Comprehensive performance comparison showing portfolio evolution, drawdowns, and metrics summary.}
\label{fig:comparison}
\end{figure}

\begin{table}[htbp]
\centering
\caption{Risk-adjusted metrics comparison}
\label{tab:risk_adjusted}
\begin{tabular}{@{}lrrr@{}}
\toprule
\textbf{Metric} & \textbf{DDPG} & \textbf{PPO} & \textbf{SPY} \\
\midrule
Sharpe Ratio & 1.78 & \textbf{1.84} & 0.89 \\
Sortino Ratio & 2.87 & \textbf{2.97} & 1.12 \\
Calmar Ratio & 2.38 & \textbf{2.48} & 0.52 \\
\bottomrule
\end{tabular}
\end{table}

\subsection{Performance Summary}

\textbf{Key Findings:}
\begin{itemize}
    \item Both agents achieve the $<$10\% maximum drawdown target with similar performance
    \item PPO slightly outperforms DDPG in risk-adjusted returns (Sharpe: 1.84 vs 1.78)
    \item Both agents maintain consistent volatility around 11\% through volatility targeting
    \item The unified hyperparameter configuration demonstrates fair algorithmic comparison
    \item Both agents significantly outperform passive benchmarks (SPY Sharpe: 0.89)
\end{itemize}

\subsection{Statistical Significance}

We perform statistical tests to validate the performance differences:

\begin{table}[htbp]
\centering
\caption{Statistical significance tests}
\label{tab:significance}
\begin{tabular}{@{}llr@{}}
\toprule
\textbf{Test} & \textbf{Comparison} & \textbf{p-value} \\
\midrule
t-test (returns) & DDPG vs PPO & $< 0.001$ \\
Wilcoxon signed-rank & DDPG vs PPO & $< 0.001$ \\
Levene's test (variance) & DDPG vs PPO & 0.034 \\
\bottomrule
\end{tabular}
\end{table}

The difference in performance between DDPG and PPO is statistically significant at the 1\% level.
