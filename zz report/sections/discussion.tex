\section{Discussion}
\label{sec:discussion}

\subsection{Why DDPG Outperforms PPO}

The substantial performance gap between DDPG and PPO can be attributed to several factors inherent to each algorithm's design and their suitability for the portfolio optimization task.

\subsubsection{Off-Policy Learning Advantage}

DDPG's off-policy nature provides a significant advantage in the financial domain:

\begin{itemize}
    \item \textbf{Sample Efficiency}: DDPG can learn from past experiences stored in the replay buffer, effectively utilizing historical market data multiple times.
    
    \item \textbf{Exploration-Exploitation Balance}: The experience replay mechanism allows DDPG to explore the action space more thoroughly while still exploiting known good strategies.
    
    \item \textbf{Diverse Learning Signal}: By sampling randomly from the replay buffer, DDPG learns from a diverse set of market conditions in each update, improving generalization.
\end{itemize}

In contrast, PPO's on-policy nature means it can only learn from its most recent experiences, potentially missing valuable lessons from earlier market regimes.

\subsubsection{Deterministic Policy Benefits}

DDPG's deterministic policy offers advantages for portfolio allocation:

\begin{itemize}
    \item \textbf{Consistency}: Given the same market state, DDPG always outputs the same allocation, leading to more stable portfolio management.
    
    \item \textbf{Interpretability}: The deterministic mapping from states to actions is easier to analyze and understand.
    
    \item \textbf{No Variance from Sampling}: Unlike PPO which samples from a distribution, DDPG's actions have no inherent randomness, reducing noise in portfolio construction.
\end{itemize}

\subsubsection{Q-Value Learning}

DDPG's critic network learns Q-values that estimate long-term returns:

\begin{equation}
Q(s, a) = \E\left[\sum_{t=0}^{\infty} \gamma^t r_t \mid s_0 = s, a_0 = a\right]
\end{equation}

This provides a more direct optimization signal compared to PPO's advantage estimation, which can suffer from high variance in financial environments with noisy rewards.

\subsection{PPO's Limitations in This Context}

Several factors contribute to PPO's relatively weaker performance:

\subsubsection{On-Policy Data Efficiency}

PPO discards data after each policy update, which is inefficient when:
\begin{itemize}
    \item Training data represents valuable historical market information
    \item Market regimes change slowly, making recent data less representative
    \item The environment (financial markets) is partially observable
\end{itemize}

\subsubsection{Clipping Limitations}

The clipped objective function, while providing stability, may be overly conservative:
\begin{equation}
L^{\text{CLIP}} = \min(r_t(\theta)\hat{A}_t, \text{clip}(r_t(\theta), 1-\epsilon, 1+\epsilon)\hat{A}_t)
\end{equation}

In rapidly changing market conditions, this clipping can prevent the agent from adapting quickly enough to new regimes.

\subsubsection{Stochastic Policy Drawbacks}

PPO's stochastic policy adds noise to portfolio allocations:
\begin{equation}
a \sim \mathcal{N}(\mu_\theta(s), \sigma_\theta(s))
\end{equation}

While beneficial for exploration, this can lead to:
\begin{itemize}
    \item Inconsistent portfolio weights across similar market states
    \item Higher transaction costs from unnecessary rebalancing
    \item Suboptimal hedging decisions due to action variance
\end{itemize}

\subsection{Options Hedging Insights}

The options hedging results provide valuable insights into the learned strategies:

\subsubsection{DDPG's Hedging Strategy}

DDPG learned to:
\begin{enumerate}
    \item \textbf{Anticipate Volatility}: Increase hedge ratios before volatility spikes, suggesting learned patterns in market behavior
    
    \item \textbf{Cost-Benefit Analysis}: Maintain hedges only when the expected protection value exceeds premium costs
    
    \item \textbf{Dynamic Adjustment}: Vary hedge ratios based on portfolio composition and market conditions
\end{enumerate}

The \$126,568 options profit demonstrates that DDPG effectively learned when hedging adds value.

\subsubsection{PPO's Conservative Approach}

PPO's lower hedge utilization (23 days vs. 89 days) suggests:
\begin{itemize}
    \item Less confidence in timing hedging decisions
    \item Preference for lower-cost strategies (minimal hedging)
    \item Possible underfitting to the hedging component of the action space
\end{itemize}

\subsection{Stop-Loss Mechanism Effectiveness}

The tiered stop-loss system proved effective for both agents:

\begin{table}[htbp]
\centering
\caption{Stop-loss trigger statistics}
\label{tab:stoploss_triggers}
\begin{tabular}{@{}lrr@{}}
\toprule
\textbf{Threshold} & \textbf{DDPG Triggers} & \textbf{PPO Triggers} \\
\midrule
5\% (reduce to 75\%) & 3 & 7 \\
10\% (reduce to 50\%) & 1 & 4 \\
15\% (reduce to 25\%) & 0 & 2 \\
\bottomrule
\end{tabular}
\end{table}

DDPG's fewer stop-loss triggers indicate:
\begin{itemize}
    \item Better risk management before reaching thresholds
    \item More effective use of options hedging for downside protection
    \item Superior positioning during market stress
\end{itemize}

\subsection{Limitations and Considerations}

\subsubsection{Data Limitations}

\begin{itemize}
    \item \textbf{Single Test Period}: Results are from one test period (2019-2020); performance may vary in other market regimes
    
    \item \textbf{Survivorship Bias}: Asset selection based on current knowledge may introduce bias
    
    \item \textbf{Transaction Costs}: Simplified transaction cost model may underestimate real-world costs
\end{itemize}

\subsubsection{Model Limitations}

\begin{itemize}
    \item \textbf{Hyperparameter Sensitivity}: Results depend on hyperparameter choices; extensive tuning on test data could lead to overfitting
    
    \item \textbf{Market Impact}: Models assume no market impact from trading, which may not hold for large portfolios
    
    \item \textbf{Partial Observability}: The state representation may not capture all relevant market information
\end{itemize}

\subsubsection{Options Model Limitations}

\begin{itemize}
    \item \textbf{Black-Scholes Assumptions}: The pricing model assumes constant volatility and log-normal returns
    
    \item \textbf{Execution Assumptions}: Perfect execution at theoretical prices may not be achievable in practice
    
    \item \textbf{Liquidity}: Options on some portfolio constituents may have limited liquidity
\end{itemize}

\subsection{Practical Implications}

For practitioners considering DRL-based portfolio management:

\subsubsection{Algorithm Selection}

\begin{itemize}
    \item \textbf{DDPG preferred} for continuous allocation tasks with stable environments
    \item \textbf{PPO may be preferred} when policy stability is paramount or in more volatile environments requiring frequent adaptation
\end{itemize}

\subsubsection{Risk Management}

\begin{itemize}
    \item Options hedging adds significant value during tail risk events
    \item Tiered stop-loss provides systematic downside protection
    \item Combining multiple risk management tools is more effective than relying on any single approach
\end{itemize}

\subsubsection{Implementation Considerations}

\begin{itemize}
    \item Extensive backtesting across multiple market regimes is essential
    \item Real-time monitoring and human oversight remain important
    \item Regular model retraining may be necessary as market dynamics evolve
\end{itemize}

\subsection{Asset Correlation Analysis}

Figure \ref{fig:correlation} shows the correlation structure of our asset universe during the training period (2010-2018).

\begin{figure}[htbp]
\centering
\includegraphics[width=0.85\textwidth]{figures/correlation_matrix.png}
\caption{Asset correlation matrix computed from training period data (2010-2018). Technology stocks show high intra-sector correlation, while bonds (TLT, AGG) provide diversification benefits through low/negative correlation with equities.}
\label{fig:correlation}
\end{figure}

Key observations from the correlation structure:
\begin{itemize}
    \item \textbf{Technology Cluster}: AAPL, MSFT, GOOGL, NVDA, AMZN show correlations of 0.6-0.8, limiting diversification within sector
    \item \textbf{Bond Diversification}: TLT and AGG have near-zero or negative correlations with equities, providing true diversification
    \item \textbf{Gold Hedge}: GLD shows low correlation ($\sim$0.1-0.2) with most assets, serving as a crisis hedge
    \item \textbf{Index Correlation}: SPY, QQQ, and IWM are highly correlated with individual stocks, providing less diversification than expected
\end{itemize}

The DRL agents learned to exploit these correlation patterns---DDPG's shift to bonds during the COVID-19 crash demonstrates understanding of the negative correlation between bonds and equities during risk-off periods.

\subsection{Comparison with Literature}

Our results are consistent with findings in the literature:

\begin{itemize}
    \item \textbf{Jiang et al. (2017)}: Reported similar advantages of deep learning approaches over traditional methods
    
    \item \textbf{Liang et al. (2018)}: Found DDPG effective for portfolio optimization on Chinese markets
    
    \item \textbf{Yang et al. (2020)}: FinRL framework shows comparable performance characteristics
\end{itemize}

However, our contribution extends the literature by:
\begin{enumerate}
    \item Integrating options-based hedging within the DRL framework
    \item Evaluating performance during a specific tail risk event (COVID-19)
    \item Providing detailed comparison between DDPG and PPO for portfolio optimization
\end{enumerate}
